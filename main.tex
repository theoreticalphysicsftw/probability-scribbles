\documentclass[a4paper]{amsart}
 
\newtheorem{deff}{Definition}
\newtheorem{prop}{Proposition}
\newtheorem{lem}{Lemma}
\newtheorem{thm}{Theorem}
\newtheorem*{notation}{Notation}

\usepackage{mathrsfs}


\newcommand{\pow}[1]{\mathscr{P}\left(#1\right)}
\newcommand{\borr}[2]{\mathscr{B}_{#1}\left(#2\right)}
\newcommand{\bor}[1]{\mathscr{B}\left(#1\right)}
\newcommand{\pair}[2]{\langle #1, #2\rangle}
\newcommand{\sig}[1]{\mathfrak{Sig}\left(#1\right)}
\newcommand{\NN}{\mathbb{N}}
\newcommand{\convg}{\uparrow}

\renewcommand{\thefootnote}{\fnsymbol{footnote}}

\begin{document}
	
\author{Mihail Mladenov}
\title{Probability scribbles}
\date{\today}
 	
\begin{abstract}
    This text is intended to condense my own knowledge in 
    probability theory and is not optimized for reading by other 
    people. The main goal is for me to gain understanding of 
    some simple facts in probability theory.
\end{abstract}

\maketitle

\tableofcontents
\newpage

\section{Preface}

Throughout this text the $ZFC$ axioms are assumed and 
first-order predicate logic is used as meta-language. Many 
results in set theory and topology will be used without 
providing proof. Throughout this book the formal language is 
freely mixed with informal/abusive/bad/hand-wavy English 
and symbolic notation. Although the goal has been to 
minimize hand-wavy bullshit there's still a lot of crap. 

Since this is not intended to be directly read by other people 
there's no need to clarify a lot of the notation used (which is fairly standard),
except in the cases where there's serious danger that I might forget 
what I meant by using that notation in which case 
explanation should be provided.

The definition-theorem-proof style is used in writing this text.


\section{Measure Theory Basics}

\subsection{Measurable spaces}

\begin{deff}
    Let $F$ be a set. A nonempty set $\mathcal{F} \subseteq 
    \pow{E}$ is called $\sigma$-algebra on $F$ if
    
    \begin{itemize}
        \item $\left(\forall A \in \mathcal{F}\right)  \left( F 
        \setminus B \in \mathcal{F} \right)$
        \item $\left(\forall A \in \mathcal{F}^\NN\right) \left( 
        \bigcup\limits_{n\in \NN} A_n \in \mathcal{F} \right) $   
        \footnote{We will often write $A_n$ instead of $A(n)$ 
        for $n \in \NN$}
    \end{itemize}
\end{deff}

\begin{prop}
    If $\mathcal{F}$ is $\sigma$-algebra on $F$, then $\emptyset \in \mathcal{F}$ and
    $F \in \mathcal{F}$.
\end{prop}

\begin{proof}
    Since $\mathcal{F}$ is $\sigma$-algebra it's nonempty. 
    Now let $A \in \mathcal{F}$ then it's true that 
    $ F = A \bigcup \left(F \setminus A\right) \in \mathcal{F}$ 
    and $\emptyset = F \setminus F \in \mathcal{F}$. 
\end{proof}

\begin{deff}
    We say that a sequence $A \in \mathcal{F}^\NN$  is increasing with limit $A_\infty$ (denoted by $A \convg A_\infty$) if
    \begin{itemize}
        \item $\left( \forall n \in \NN\right) \left( A_n \subseteq A_{n+1} \right)$
        \item $\bigcup\limits_{n\in \NN} A_n = A_\infty$
    \end{itemize}
\end{deff}

\begin{deff}
    If $F$ is a set we denote the set of all $\sigma$-algebras over $F$ by $\sig{F}$.
\end{deff}
\begin{deff}
    The ordered pair $\pair{F}{\mathcal{F}}$ is called measurable space if $\mathcal{F} \in \sig{F}$.
\end{deff}

\begin{deff}
    Let $F$ be a set. A set $\mathcal{F} \subseteq \pow{E}$ is called $d$-system on $F$ if 
    \begin{itemize}
        \item $ F \in \mathcal{F}$
        \item $\left(\forall A \in \mathcal{F}\right) \left(\forall B \in \mathcal{F}\right)
        \left( A \subseteq B \Longrightarrow B \setminus A \in \mathcal{F} \right)$
        \item $\left(\forall A \in \mathcal{F}^\NN \right) \left( A \convg A_\infty \Longrightarrow A_\infty \in \mathcal{F} \right)$
    \end{itemize}
\end{deff}

\begin{deff}
    Let $F$ be a set. A set $\mathcal{F} \subseteq \pow{E}$ is 
    called $\pi$-system on $F$ if 
    
    $$ \left(\forall A \in \mathcal{F}\right) \left(\forall B \in 
    \mathcal{F}\right) \left( A \cap B \in \mathcal{F}\right) $$
\end{deff}

\begin{prop}
    Let $F$ be a set. A set $\mathcal{F} \subseteq \pow{E}$ is 
    $\sigma$-algebra on $F$ if and only if it's both $\pi$ and 
    $d$ system.
\end{prop}

\begin{proof}
    Assuming that $\mathcal{F}$ is $\sigma$-algebra on $F$,
    we can immediately infer that it's $\pi$ system using the 
    countable union property and the complement property of 
    the $\sigma$-algebra together with De Morgan's law. It's 
    also immediate that it's $d$ system.
    
    Now assume that $\mathcal{F}$ is $\pi$ and $d$ system. We have that $F \in \mathcal{F}$
    and since
    $$\left(\forall A \in \mathcal{F}\right) \left(\forall B \in \mathcal{F}\right)
    \left( A \subseteq B \Longrightarrow B \setminus A \in\mathcal{F} \right)$$
    we have that
    $$\left(\forall B \in \mathcal{F}\right)  \left( F \setminus B \in \mathcal{F} \right)$$
    Now let $S\in \mathcal{F}^\NN$ be a sequence. 
    Since a $\pi$ system is closed under intersection and 
    furthermore we've proven that 
    $\mathcal{F}$ is closed under complements it's also 
    closed under finite number of unions using De Morgan's 
    law and trivial induction. Now we construct the sequence $\tilde{S}\in \mathcal{F}^\NN$
    $$
    \tilde{S}_n := \bigcup\limits_{k\in (n + 1)} S_k
    $$
    Now we have that 
    $ \left(\forall n \in \NN\right) \left( \tilde{S}_n  \subseteq \tilde{S}_{n+1} \right)$
    and so it follows that
    $\bigcup\limits_{n\in \NN} \tilde{S}_n \in \mathcal{F}$.
    On the other hand we have
    $$
    \mathcal{F} \ni \bigcup\limits_{n\in \NN} \tilde{S}_n = \bigcup\limits_{n\in \NN} S_n 
    $$
    From the things proven so far we can conclude that $\mathcal{F}$ is $\sigma$-algebra.
\end{proof}

\begin{prop}
    Let $F$ be a set. Let $\mathfrak{F} \subseteq \pow{\pow{F}}$ be collection of $\sigma$-algebras over 
    $F$. Then the intersection of this collection $\bigcap \mathfrak{F}$ is again $\sigma$-algebra.
\end{prop}

\begin{proof}
    Let $A \in \bigcap \mathfrak{F}$. From this we can conclude that 
    $\left(\forall \mathcal{F} \in \mathfrak{F}\right) \left(A \in \mathcal{F}\right) $ and since all of them are 
    $\sigma$-algebras we have 
    $\left(\forall \mathcal{F} \in \mathfrak{F}\right) \left(F \setminus A \in \mathcal{F}\right)$ 
    from which we immediately conclude that $F \setminus A \in \bigcap \mathfrak{F}$.
    
    Now let $S \in \left(\bigcap\mathfrak{F}\right)^\NN$ be a 
    sequence in $\bigcap\mathfrak{F}$. Since for every 
    $\mathcal{F} \in \mathfrak{F}$ we have that
    $\bigcap\mathfrak{F} \subseteq \mathcal{F}$ it follows 
    that $S$ is sequence in every $\mathcal{F}$. From this we 
    conclude that 
    $\bigcup\limits_{n\in \NN} S_n \in \mathcal{F}$ for every $\mathcal{F}$ 
    since it's $\sigma$-algebra. 
    But then  $ \bigcup\limits_{n\in \NN} S_n \in  \bigcap \mathfrak{F}$.
\end{proof}

\begin{deff}
    Let $F$ be a set and $\mathcal{C} \subseteq \pow{F}$. We 
    define the $\sigma$-algebra generated by $\mathcal{C}$, 
    denoted by $\sigma\mathcal{C}$ to be the set
    $$
    \sigma\mathcal{C} := \bigcap\left\{ \mathcal{F} \in \sig{F} \mid \mathcal{C} \subseteq \mathcal{F} \right\}
    $$
\end{deff}

\begin{lem}
    Let $\mathcal{D}$ be a $d$-system on $F$ and pick $D \in \mathcal{D}$. The set 
    $$ \mathcal{D}_D := \left\{ X \in \mathcal{D} \mid X \cap D \in \mathcal{D} \right\} $$
    is also $d$-system on $F$.
\end{lem}

\begin{proof}
    Since $\mathcal{D}$ is a $d$-sysyem 
    $F \in \mathcal{D}$ and since $F \cap D = D \in \mathcal{D}$ 
    then $F \in \mathcal{D}_D$. 
    
    Now let $X,Y \in \mathcal{D}_D$ and $X \subseteq Y$.
    Since $X \cap D$ and $Y \cap D$ are in $\mathcal{D}$ we 
    have that 
    $ \left(Y \cap D\right) \setminus \left(X \cap D\right) \in \mathcal{D}$. 
    Now let $z \in  \left(Y \cap D\right) \setminus \left(X \cap D\right)$. 
    We have that $z \in Y \cap D$ and $z \notin X \cap D$. 
    From here $z \in Y$ and $z \in D$ and $z \notin X$ hence 
    $z \in \left(Y \setminus X\right) \cap D$ and so
    $
    \left(Y\cap D\right) \setminus \left(X\cap D\right) \subseteq  \left(Y \setminus X\right) \cap D
    $. If $w \in \left(Y \setminus X\right) \cap D $ then 
    $w \in D$ and $w \in Y$ and $w \notin X$. From here 
    follows that $w \in \left(Y \cap D\right) \setminus X$
    and so $w  \in \left(Y \cap D\right) \setminus \left(X \cap D \right)$.
    This proves that $ \left(Y \setminus X\right) \cap D  \subseteq\left(Y \cap D\right) \setminus \left(X \cap D \right)$.
    From these inclusions we can conclude that 
    $\left(Y \setminus X\right) \cap D  = \left(Y \cap D\right) \setminus \left(X \cap D \right)$.
    Now $\left(Y \setminus X\right) \cap D  \in \mathcal{D}$ and so $Y \setminus X \in \mathcal{D}_D$.
    
    Let $A \in \mathcal{D}_D^\NN$ be a sequence and $A \convg A_\infty$.
    Since for all $n$ it's true that $A_n \in \mathcal{D}_D$ we have that $A_n \cap D \in \mathcal{D}$.
    Now since $\mathcal{D}$ is $d$-system 
    $$ A_\infty \cap D = \left(\bigcup\limits_{k \in \NN} A_n\right) \cap D = \bigcup\limits_{k \in \NN} \left(A_n \cap D\right) \in \mathcal{D}$$
    From this we conclude that $A_\infty \in \mathcal{D}_D$.
\end{proof}


\begin{thm}[Monotone class theorem]
    If $\mathcal{D}$ is a $d$-system on $F$ and $\mathcal{C}\subseteq \mathcal{D}$
    is $\pi$-system on $F$, then $\sigma\mathcal{C} \subseteq \mathcal{D}$.
\end{thm}

\begin{proof}
    Let $\mathcal{D}_* \subseteq \mathcal{D}$ be the intersection of all $d$-systems containing  $\mathcal{C}$.
    
    If $C \in \mathcal{C}$ we define 
    $$ \mathcal{D}_*^\prime := \left\{ D \in \mathcal{D}_* \mid D \cap C \in \mathcal{D}_* \right\}$$
    By the previous lemma $\mathcal{D}_*^\prime$ is a $d$-
    system.
    Now if $X \in \mathcal{C}  \subseteq \mathcal{D}_* $ 
    since $\mathcal{C}$ is a $\pi$-system we have that 
    $X \cap C \in \mathcal{C} \subseteq \mathcal{D}_*$ 
    which demonstrates that $X \in  \mathcal{D}_*^\prime$. 
    Since $\mathcal{D}_*^\prime$ is a $d$-system containing 
    $\mathcal{C}$ it must contain the intersection of all such 
    systems namely $\mathcal{D}_*$, but since it is itself 
    contained in $\mathcal{D}_*$ they must coincide. This 
    demonstrates that for every $C \in \mathcal{C}$ and for 
    every $D \in \mathcal{D}_*$ it's true that $C \cap D \in \mathcal{D}_*$.
    
    Now let's pick $X \in \mathcal{D}_*$ and define the set
    $$
    \mathcal{D}_*^{\prime\prime} := \left\{ D \in \mathcal{D}_* \mid D \cap X \in \mathcal{D}_* \right\}
    $$
    This set is a $d$-system by the previous lemma.
    Furthermore it contains $\mathcal{C}$ since for every $C \in \mathcal{C}$ and for 
    every $D \in \mathcal{D}_*$ it's true that $C \cap D \in \mathcal{D}_*$. From this we can conclude that 
    $\mathcal{D}_* \subseteq \mathcal{D}_*^{\prime\prime}$ 
    but since $\mathcal{D}_*^{\prime\prime} \subseteq \mathcal{D}_*$ 
    by construction these sets coincide. But now this means 
    that for every $D, X \in \mathcal{D}_*$ we have $D \cap X \in \mathcal{D}_*$ which means that $\mathcal{D}_*$ is
    a $\pi$-system and since it's $d$-system at the same time
    it's $\sigma$-algebra.
\end{proof}




\subsection{Measurable functions}


\begin{deff}
    If $\pair{E}{\mathcal{E}}$ and $\pair{F}{\mathcal{F}}$ are 
    measurable spaces, then a map $f : E \longrightarrow F$ is 
    said to be measurable with respect to $\mathcal{E}$ and $\mathcal{F}$ if 
    $$
    \left(\forall X \in \mathcal{F}\right) \left( f^{-1}[X] \in \mathcal{F} \right)
    $$
\end{deff}

\begin{prop}
    Let $\pair{E}{\mathcal{E}}$, $\pair{F}{\mathcal{F}}$ and
    $\pair{G}{\mathcal{G}}$ be measurable spaces. If the 
    functions $f : E \longrightarrow F$ and
    $g: F \longrightarrow G$ are measurable then $g \circ f$ is 
    measurable as well.
\end{prop}

\begin{proof}
    Let's pick $X \in \mathcal{G}$. We have that $ g^{-1}[X] \in \mathcal{F}$ since $g$ is measurable. 
    Then $f^{-1}[g^{-1}[X] ] \in \mathcal{E}$ 
    because $f$ is measurable. Now we 
    can conclude that $g \circ f$ is measurable by noticing 
    that
    $$
    \left(g \circ f\right)^{-1}[X] = f^{-1}[g^{-1}[X] ] \in \mathcal{E}
    $$
\end{proof}

\begin{prop}
    Let  $\pair{E}{\mathcal{E}}$ and $\pair{F}{\mathcal{F}}$ be 
    measurable spaces and let $\mathcal{F}_0$ be such that
    $\sigma\mathcal{F}_0 = \mathcal{F}$. Under these 
    circumstances a function $f : E \longrightarrow F$ is
    measurable if and only if 
    $$
    \left( \forall X \in \mathcal{F}_0\right)
    \left( f^{-1}[X] \in \mathcal{E}\right)
    $$
\end{prop}

\begin{proof}
    It's immediate that if $f$ is measurable than the condition 
    is satisfied as a special case.
    
    Now let's assume that $f$ satisfies the condition. We 
    define the set
    $$
    \mathcal{F}_* := \left\{ X \in \mathcal{F} \mid  f^{-1}[X] \in \mathcal{E} \right\}
    $$
    The goal is to show that $\mathcal{F}_*$ is
    $\sigma$-algebra, because since it contains $\mathcal{F}_0$ 
    then it contains $\sigma\mathcal{F}_0 = \mathcal{F}$ and 
    and since it is contained in $\mathcal{F}$ they coincide, 
    proving the proposition.
    
    Firstly let $X \in \mathcal{F}_*$. Since $f^{-1}[X] \in \mathcal{E}$
    and $\mathcal{E}$ is $\sigma$-algebra can conclude  that 
    the following holds
    $$
    f^{-1}[F\setminus X] = f^{-1}[F] \setminus f^{-1}[X] = E \setminus f^{-1}[X] \in \mathcal{E}
    $$
    which implies that $F\setminus X\in \mathcal{F}_*$.
    
    Now let's pick a sequence $S \in \mathcal{F}_*^\NN$ then
    $$
    f^{-1}\left[ \bigcup\limits_{n \in \NN} S_n \right] = 
    \bigcup\limits_{n \in \NN} f^{-1}[S_n] \in \mathcal{E}
    $$
    since $\left(n \in \NN \right) \left(f^{-1}[S_n] \in \mathcal{E}\right)$ and 
    $\mathcal{E}$ is $\sigma$-algebra. From this we can conclude
    that $\bigcup\limits_{n \in \NN} S_n \in \mathcal{F}_*$
    which demonstrates that $\mathcal{F}_*$ is $\sigma$-algebra.
\end{proof}



\begin{deff}
    Let $\pair{T}{\mathcal{O}}$ be a topological space, then 
    $\sigma\mathcal{O}$ is called the Borel $\sigma$-algebra 
    of $T$ with respect to $\mathcal{O}$ and we denote it by 
    $\borr{\mathcal{O}}{T}$.
\end{deff}

By abuse of notation we will often write $\bor{T}$ and infer 
the topology from the context.

\begin{prop}
    Let $\pair{T}{\mathcal{O}}$ and $\pair{\tilde{T}}{\tilde{\mathcal{O}}}$ be topological spaces. 
    If $f : T \longrightarrow \tilde{T}$ is continuous with 
    respect to $\mathcal{O}$ and $\tilde{\mathcal{O}}$ then 
    it's measurable with respect to $\bor{T}$ and $
    \bor{\tilde{T}}$.
\end{prop}

\begin{proof}
    Since $f$ is continuous for every open set $O \in 
    \tilde{\mathcal{O}}$  we have
    $$
    f^{-1}[O] \in \mathcal{O} \subseteq \sigma\mathcal{O} 
    = \bor{T}
    $$
    which proves the result by using the previous proposition.
\end{proof}



\subsection{Measure spaces}

\subsection{Integration}

\section{Basic Probability}

\subsection{Probability spaces}

\end{document}
